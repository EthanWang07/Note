\documentclass[10pt, a4paper]{article}
\usepackage[UTF8]{ctex}
\usepackage[top=3cm, bottom=4cm, left=3.5cm, right=3.5cm]{geometry}
\usepackage{amsmath,amsthm,amsfonts,amssymb,amscd, fancyhdr, color, comment, graphicx, environ} % Mathtools
\usepackage{float}
\usepackage{mathrsfs}
\usepackage[math-style=ISO]{unicode-math}
\setmathfont{Latin Modern Math}
\usepackage{anyfontsize} %字体大小
\usepackage{tikz} %画图

\begin{document}

\section*{Randomized Trials (随机实验)} 
\boxed{\text{Causal Effect (因果效应): } Y_{1i} - Y_{0i}} \\

If we want to find the causal effect of a treatment, naturally we need to compare the outcome of the treatment group and the control group.\\
First we can construct a dummy variable $D_i$ to indicate whether the $i$-th component is in the treatment group or the control group:
\[D_i = \begin{cases}
1, & \text{if the $i$-th component is in the treatment group} \\
0, & \text{if the $i$-th component is in the control group}
\end{cases}\]

\[\text{Diffenence in group means} = Avg_n[Y_{1i} \mid D_i = 1] - Avg_n[Y_{0i} \mid D_i = 0] \]
But this equation is not what we are looking for. (Why?) We use \kappa to denote the effect of the treatment, then we have:
\[ Y_{1i} = Y_{0i} + \kappa \]
\begin{align*}
    \text{Diffenence in group means} &= Avg_n[Y_{1i} \mid D_i = 1] - Avg_n[Y_{0i} \mid D_i = 0] \\
    &= {\kappa + Avg_n[Y_{0i} \mid D_i = 1]} - Avg_n[Y_{0i} \mid D_i = 0] \\
    &= \kappa + \{Avg_n[Y_{0i} \mid D_i = 1] - Avg_n[Y_{0i} \mid D_i = 0]\}
\end{align*}
According to this, the difference between the two groups can be written as:
\[\boxed{\text{Diffenence in group means} = \text{Average causal effect} + \text{Section bias (选择偏误)}}\]  

\subsection*{统计推断基础}
样本均值的无偏性:$E[\bar{Y}] = E[Y_i]


\end{document}